\documentclass[a4paper,twoside,12pt,openright]{report}

\usepackage{ifthen}
\usepackage{a4wide}
\usepackage[charter]{mathdesign}
\usepackage[T1]{fontenc}
\usepackage[scaled=0.88]{beramono}
\usepackage{amsmath}
\usepackage{amsthm}
\usepackage{color}
\usepackage{pifont}
\usepackage[dvips]{graphicx}
\usepackage{pstricks}
\usepackage{pstricks-add}
\usepackage{pst-node}
\usepackage{picinpar}
\usepackage{subfigure}
\usepackage{calc}
\usepackage{colortbl}

\definecolor{GecodeGreen}{rgb}{0.04314,0.46275,0.27451}
\definecolor{GecodeBlue}{rgb}{0,0.36078,0.631273}
\definecolor{GecodeRed}{rgb}{0.8549,0.1451,0.1137255}
\definecolor{GecodeOrange}{rgb}{0.9216,0.537255,0.10588}

\usepackage[ps2pdf,%
            breaklinks=false,
            hyperindex=true,
            bookmarksnumbered=true,
            colorlinks=true,
            linkcolor=GecodeBlue,
            pagecolor=GecodeBlue,
            citecolor=GecodeBlue,
            filecolor=GecodeGreen,
            urlcolor=GecodeGreen,
            pdftitle={\PDFTITLE},
            pdfauthor={\PDFAUTHOR},
            pdfsubject={\PDFSUBJECT},
            pdfkeywords={\PDFKEYWORDS},
            pdfcreator={},
            pdfproducer={},
            pdfdisplaydoctitle=true
            ]{hyperref}

% Redefine math tt alphabet to use beromono
\SetMathAlphabet{\mathtt}{normal}{\encodingdefault}{\ttdefault}{m}{n}%
\SetMathAlphabet{\mathtt}{bold}{\encodingdefault}{\ttdefault}{b}{n}%

\theoremstyle{definition}
\newtheorem{rawtip}{Tip}[chapter]

\newcommand{\rawtipautorefname}{Tip}
\renewcommand{\sectionautorefname}{Section}
\renewcommand{\subsectionautorefname}{Section}
\renewcommand{\chapterautorefname}{Chapter}

\def\CPP{\leavevmode\textrm{\hbox{C\hskip
-0.1ex\raise 0.5ex\hbox{\tiny ++}}}}

%% Absolute URL
\newcommand{\AURL}[2]{\href{#1}{#2}}
%% Relative URL
\newcommand{\RURL}[2]{\href{http://www.gecode.org/doc/\GecodeVersion/#1}{#2}}

\renewcommand{\labelitemi}{\mbox{\raisebox{1.4pt}{\tiny\ding{110}}}}

\newcommand{\DOWNLOAD}[2]{\RURL{MPG/#1}{#2}}

\newcommand{\CppInline}[1]{\texttt{#1}}
\newcommand{\litanon}{$\cdots$}
\newcommand{\litkw}[1]{\textbf{#1}}
\newcommand{\litkwt}[1]{\CppInline{\textbf{#1}}}
\newcommand{\litc}[1]{\textit{#1}}
\newcommand{\lits}{\mbox{ }}
\newcommand{\litstr}[1]{\textcolor{GecodeGreen}{#1}}
\newcommand{\litlabel}[2]{\hypertarget{#1}{\textcolor{GecodeRed}{\normalfont\bfseries\textsc{#2} $\equiv$}}}
\newcommand{\litfile}[3]{\hypertarget{#1}{\textcolor{GecodeRed}{\normalfont\bfseries\textsc{#2}
      $\equiv$}}\hfill% 
\textcolor{GecodeBlue}{\normalfont\bfseries\textsc{%
\DOWNLOAD{#3}{[download]}}}}
\newcommand{\litref}[2]{\hyperlink{#1}{$\blacktriangleright$\,\normalfont\bfseries\textsc{#2}}}

\newcommand{\fcbox}[2]{%
\psframebox[cornersize=absolute,linearc=0.05,linecolor=#1,linewidth=0.01pt,opacity=0.20,fillcolor=#1,fillstyle=solid]{#2}}

\newcommand{\litcmdblock}[1]{%
\begin{center}\fcbox{GecodeRed}{\begin{minipage}{0.95\textwidth}\ttfamily #1%
\end{minipage}}\end{center}}

\newcommand{\smalllitcmdblock}[1]{%
\begin{center}\fcbox{GecodeRed}{\begin{minipage}{0.9\textwidth}\ttfamily #1%
\end{minipage}}\end{center}}

\newcommand{\litcodeblock}[1]{%
\begin{center}\fcbox{GecodeOrange}{\begin{minipage}{0.95\textwidth}\ttfamily #1%
\end{minipage}}\end{center}}

\newcommand{\smalllitcodeblock}[1]{%
\begin{center}\fcbox{GecodeOrange}{\begin{minipage}{0.9\textwidth}\ttfamily #1%
\end{minipage}}\end{center}}

\newcommand{\litcodepage}[2]{%
\fcbox{GecodeOrange}{\begin{minipage}{#2}\ttfamily #1\end{minipage}}}

\newcommand{\gecodedocbaseref}[2]{\RURL{reference/#1}{#2}}
\newcommand{\litdocref}[2]{\gecodedocbaseref{#1}{#2}}

\newcommand{\otherref}[2]{\autoref{#1}~(#2)}

\newcommand{\publink}[1]{\hfill\AURL{#1}{[download]}}

\newcommand{\fixme}[1]{\textbf{Fixme:} #1}


\newcommand{\tip}[2]{%
\begin{rawtip}[#1]%
\addcontentsline{lot}{table}
{\protect\numberline{\therawtip}#1}%
#2\hfill$\blacktriangleleft$\end{rawtip}%
}

\newenvironment{convention}%
{\paragraph{Convention.}}{}
\newenvironment{important}%
{\paragraph{Important.}}{}

\newcommand{\YES}{{\ding{51}}}

\newcommand{\setc}[2]{\{#1\;|\;#2\}}

\newcommand{\fcframe}[2]{%
\psframe[cornersize=absolute,linearc=0.05,linecolor=white,linewidth=0.0,fillcolor=white,fillstyle=solid]#2%
\psframe[cornersize=absolute,linearc=0.05,linecolor=#1,linewidth=0.01pt,opacity=0.50,fillcolor=#1,fillstyle=solid]#2%
}
\newcommand{\fcbwframe}{%
\psframe[cornersize=absolute,linearc=0.05,linecolor=black,linewidth=0.25pt,fillcolor=white,fillstyle=solid]}
\newcommand{\fccircle}[1]{%
\pscircle[linecolor=#1,linewidth=0.01pt,opacity=0.50,fillcolor=#1,fillstyle=solid]}


%%%
%%% Chapter and part headings
%%%
\makeatletter

\newcommand{\MyBigNumber}{\fontsize{1in}{1in}\selectfont}
\newcommand{\MyBigTitle}{\fontsize{.45in}{.45in}\selectfont}

\def\@part[#1]#2{%
    \ifnum \c@secnumdepth >-2\relax
      \refstepcounter{part}%
      \addcontentsline{toc}{part}{\thepart\hspace{1em}#1}%
    \else
      \addcontentsline{toc}{part}{#1}%
    \fi
    \markboth{}{}%
    \thispagestyle{empty}%
    {\raggedleft
     \interlinepenalty \@M
     \normalfont
     \ifnum \c@secnumdepth >-2\relax
        \psframebox[linewidth=.1pt,linecolor=GecodeOrange,framearc=.05,framesep=.2in,fillcolor=GecodeOrange,opacity=0.2,fillstyle=solid,boxsep=false]{\bfseries\MyBigNumber\thepart}
       \vskip 60\p@
     \fi
    \interlinepenalty\@M
    {\raggedleft\MyBigTitle\bfseries #2}
    \par\bigskip\par
    {\raggedleft\large\bfseries\partauthors}
    \par}%
    \@endpart}
\def\@spart#1{%
    {\centering
     \interlinepenalty \@M
     \normalfont
     \Huge \bfseries #1\par}%
    \@endpart}
\def\@endpart{\vfil\newpage
              \if@twoside
               \if@openright
                \null
                \thispagestyle{empty}%
                \partblurb\newpage
               \fi
              \fi
              \if@tempswa
                \twocolumn
              \fi}

\newcommand{\partauthors}{}
\newcommand{\partblurb}{}
\newcommand{\partfooter}{}
\renewcommand\thepart{\Alph{part}}
\newcommand{\mypart}[2]{
\setcounter{part}{#1}\part{#2}%
}

\newlength{\MyChapterNumberWidth}
\newlength{\MyChapterRestWidth}

\def\@makechapterhead#1{%
  \vspace*{20\p@}%
  {\parindent \z@ \raggedright \normalfont
    \ifnum \c@secnumdepth >\m@ne
        \settowidth{\MyChapterNumberWidth}{\bfseries\MyBigNumber\thechapter}
        \setlength{\MyChapterRestWidth}{\textwidth}
        \addtolength{\MyChapterRestWidth}{-\MyChapterNumberWidth}
        \addtolength{\MyChapterRestWidth}{-40pt}
        \parbox[t]{\MyChapterNumberWidth}{\psframebox[linewidth=.1pt,linecolor=GecodeBlue,framearc=.05,framesep=.2in,fillcolor=GecodeBlue,opacity=0.2,fillstyle=solid,boxsep=false]{\bfseries\MyBigNumber\thechapter}}
    \fi
    \interlinepenalty\@M
    \hfill\parbox[b]{\MyChapterRestWidth}{\raggedleft\MyBigTitle\bfseries #1}
    \par\nobreak
    \vskip 40\p@
  }}

\def\@makeschapterhead#1{%
  \vspace*{20\p@}%
  {\parindent \z@ \raggedright
    \normalfont
    \interlinepenalty\@M
    \MyBigTitle\bfseries  #1\par\nobreak
    \vskip 40\p@
  }}

\makeatother

\newenvironment{changelog}{\begin{itemize}\itemsep 0pt}{\end{itemize}}
\newcommand{\cle}[3][none]{%
\item #3 (#2\ifthenelse{\equal{#1}{none}}{}{, thanks to #1}).}


%%% Node stuff

\newcommand{\DefineNode}[3]{\pnode(#1,#2){#3}}

\newcommand{\FailedNode}[2]{%
\rput(#1,#2){\psframe[fillcolor=GecodeRed,fillstyle=solid](-8,-8)(8,8)}}
\newcommand{\FailedNodeI}[3]{%
\rput(#1,#2){\psframe[fillcolor=GecodeRed,fillstyle=solid](-8,-8)(8,8)\makebox(0,0){#3}}}
\newcommand{\SolvedNode}[2]{%
\rput(#1,#2){\pspolygon[fillcolor=GecodeGreen,fillstyle=solid](-10,0)(0,-10)(10,0)(0,10)}}
\newcommand{\SolvedNodeI}[3]{%
\rput(#1,#2){\pspolygon[fillcolor=GecodeGreen,fillstyle=solid](-10,0)(0,-10)(10,0)(0,10)\makebox(0,0){#3}}}

\newgray{MyULG}{0.9}

\newcommand{\ChoiceNode}[2]{%
\rput(#1,#2){\pscircle[fillcolor=GecodeBlue,fillstyle=solid]{10}}}
\newcommand{\ChoiceNodeI}[3]{%
\rput(#1,#2){\pscircle[fillcolor=GecodeBlue,fillstyle=solid]{10}\makebox(0,0){#3}}}
\newcommand{\UChoiceNode}[2]{%
\rput(#1,#2){\pscircle[fillcolor=white,fillstyle=solid]{10}}%
\rput(#1,#2){\pscircle[fillcolor=GecodeBlue,fillstyle=solid,opacity=0.4]{10}}}
\newcommand{\UChoiceNodeI}[3]{%
\rput(#1,#2){\pscircle[fillcolor=white,fillstyle=solid]{10}}%
\rput(#1,#2){\pscircle[fillcolor=GecodeBlue,fillstyle=solid,opacity=0.4]{10}\makebox(0,0){#3}}}
\newcommand{\RootNode}[2]{%
\rput(#1,#2){\pscircle[fillcolor=GecodeOrange,fillstyle=solid]{10}}}
\newcommand{\RootNodeI}[3]{%
\rput(#1,#2){\pscircle[fillcolor=GecodeOrange,fillstyle=solid]{10}\makebox(0,0){#3}}}
\newcommand{\GoneNode}[2]{%
\rput(#1,#2){\pscircle[fillcolor=MyULG,fillstyle=solid]{10}}}


\newcommand{\GuessNode}[2]{%
\rput(#1,#2){\pscircle[fillcolor=MyULG,fillstyle=solid]{10}\makebox(0,0){?}}}


\newcommand{\DefineLink}[2]{\ncline{-}{#1}{#2}}
\newcommand{\DefineFatLink}[2]{\ncline[linewidth=3pt]{-}{#1}{#2}}
\newcommand{\HiddenLinkL}[3]{\ncline[linewidth=0pt,linecolor=white]{-}{#1}{#2}\mput{\mbox{\strut\qquad\qquad #3}}}
\newcommand{\HiddenLinkR}[3]{\ncline[linewidth=0pt,linecolor=white]{-}{#1}{#2}\mput{\mbox{\strut#3\qquad\qquad}}}
\newcommand{\DefineLinkA}[2]{\ncline{->}{#1}{#2}}


%% Macros for nonterminal and expressions, etc
\newcommand{\NONTERM}[1]{\langle#1\rangle}
\newcommand{\NINT}{\NONTERM{n}}
\newcommand{\NVAR}{\NONTERM{x}}
\newcommand{\NSET}{\NONTERM{s}}
\newcommand{\NVARS}{\NONTERM{\overline x}}
\newcommand{\NINTS}{\NONTERM{\overline n}}
\newcommand{\NREL}{\NONTERM{r}}
\newcommand{\NLE}{\NONTERM{\mathit{IntExpr}}}
\newcommand{\NLR}{\NONTERM{\mathit{IntRel}}}
\newcommand{\NBE}{\NONTERM{\mathit{BoolExpr}}}
\newcommand{\NBR}{\NONTERM{\mathit{BoolRel}}}
\newcommand{\NSE}{\NONTERM{\mathit{SetExpr}}}
\newcommand{\NSR}{\NONTERM{\mathit{SetRel}}}


\newcommand{\reify}[2]{#1\Leftrightarrow #2=\mathtt 1}

\newcommand{\boundsz}{\ensuremath{\operatorname{bounds}(\mathbb{Z})}}
\newcommand{\range}[2]{\left[#1\;..\;#2\right]}
\newcommand{\ranges}[1]{\operatorname{ranges}(#1)}
\newcommand{\seq}[1]{\left\langle#1\right\rangle}
\newcommand{\seqc}[3]{\left\langle#1\right\rangle_{#2}^{#3}}
\newcommand{\NN}{\ensuremath{\mathbb N}}
\newcommand{\ZZ}{\ensuremath{\mathbb Z}}
\newcommand{\RR}{\ensuremath{\mathbb R}}

\newcommand{\rseqc}[3]{\seqc{\range{{#1}_{i}}{{#2}_{i}}}{i=0}{#3}}


%% KNIGHTS board

\newcommand{\knsolidfield}{%
\psframe[fillstyle=solid,fillcolor=GecodeBlue,opacity=0.3,linecolor=GecodeBlue,linewidth=0pt,dimen=inner](0,0)(2,2)%
}
\newcounter{knsize}%
\newcounter{knlines}%
\newcounter{knrepeat}%
\newcommand{\knboard}[1]{
\setcounter{knsize}{#1*2}%
\setcounter{knlines}{#1+1}%
\setcounter{knrepeat}{#1/2}%
\begin{pspicture}(0,0)(\value{knsize},\value{knsize})
\multiput(0,0)(0,4){\value{knrepeat}}{%
  \multiput(0,0)(4,0){\value{knrepeat}}{\knsolidfield}
  \multiput(2,2)(4,0){\value{knrepeat}}{\knsolidfield}
}
\multiput(0,0)(2,0){\value{knlines}}{%
\psline[linecolor=GecodeBlue,linewidth=1.5pt,dimen=inner](0,0)(0,\value{knsize})}
\multiput(0,0)(0,2){\value{knlines}}{%
\psline[linecolor=GecodeBlue,linewidth=1.5pt,dimen=inner](0,0)(\value{knsize},0)}
}
\newcounter{knx}\newcounter{kny}%
\newcommand{\knfield}[4]{%
\setcounter{knx}{2*#3+1}\setcounter{kny}{2*#4+1}%
\rput(\value{knx},\value{kny}){\rnode{n#1}{%
\textcolor{GecodeBlue}{#2}}}%
}
\newcommand{\knmove}[3]{%
\ncline[linewidth=1.8pt,nodesep=3pt]{->}{n#2}{n#3}%
}

%% CSP lib

\newcommand{\CSPLIB}[1]{Problem~#1 in
  \AURL{http://www.csplib.org/}{CSPLib}}
